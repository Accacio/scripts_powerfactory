\documentclass[a4paper,twoside]{articlewithlogo}

\usepackage{enumerate}
%\usepackage{enumitem}
\usepackage{graphicx}
\graphicspath{{Figuras/}}
\usepackage{color}
\usepackage[cmex10]{amsmath}
\usepackage{array}
\usepackage{float}
\usepackage{multicol}
\usepackage[utf8]{inputenc} 
\usepackage[T1]{fontenc}
\usepackage[french]{babel}
%\usepackage[font=normalsize,format=plain,labelfont=bf,up,textfont=up,figurename=Figura,tablename=Tabela]{caption}
\usepackage{subcaption}
\usepackage[top=1in, bottom=1in, left=1.25in, right=1.25in]{geometry}
\usepackage{indentfirst}
\usepackage{fancyhdr}
% Font packages
\usepackage{amssymb}
\usepackage{amsfonts}
\usepackage{pgfgantt}
\usepackage{steinmetz}
% Nice extra font package, e.g. \mathds{1}
\usepackage{dsfont}
\usepackage{color}
\usepackage{blindtext}
% Use multiple rows when writing tables
\usepackage{multirow}
\usepackage{booktabs}
\usepackage{bm}
\usepackage{bigstrut}
% Uncomment next line to make footnots per page
\usepackage{perpage}
% Uncoment next group of lines to create the table of contents for the PDF
\usepackage{hyperref}
\usepackage[toc,page]{appendix}

\definecolor{darkblue}{rgb}{0,0,0.5}
\definecolor{darkblue}{rgb}{0,0,0.5}
\renewcommand{\title}{Étude des régulations en tension des réseaux de distribution}
\newcommand{\subtitle}{Rapport d'activité de Stage 2A}

\hypersetup{
    pdftitle={\title},
    pdfauthor={Rafael Accacio Nogueira},
    bookmarksnumbered=true,     
    bookmarksopen=true,         
    bookmarksopenlevel=1,       
    colorlinks=true,
    linkcolor=black,
    filecolor=darkblue,  
    urlcolor=darkblue,  
    citecolor=darkblue,              
    pdfstartview=Fit,          
    pdfpagemode=UseOutlines,    % this is the option you were lookin for
    pdfpagelayout=TwoPageRight
}
\let\oldcontentsline\contentsline%
\renewcommand\contentsline[4]{%
    \oldcontentsline{#1}{\smash{\raisebox{1em}{\hypertarget{toc#4}{}}}#2}{#3}{#4}}

\newcommand\mysection[1]{\section[#1]{\protect\hyperlink{tocsection.\thesection}{#1}}\label{#1}}
\newcommand\mysubsection[1]{\subsection[#1]{\protect\hyperlink{tocsection.\thesection}{#1}}\label{#1}}
\newcommand\mysubsubsection[1]{\subsubsection[#1]{\protect\hyperlink{tocsection.\thesection}{#1}}\label{#1}}

\newcommand{\conteudo}{\tableofcontents\label{tocsection}}


\pagestyle{fancy}
\newif\ifdebug
\newcommand{\draft}{\debugtrue}
\newcommand{\final}{\debugfalse}
\newcommand\todo[1]{\ifdebug {\color{red}#1}\else \PackageError{}{FORGOT TO DO SOMETHING}{}\fi}
\newcommand\doing[1]{\ifdebug {\color{blue}#1}\fi}
\newcommand\warning[1]{\ifdebug {\color{red}#1}\fi}


\fancyhead[CO]{\title}
\fancyhead[CE]{\subtitle}
%\fancyhead[RE]{\rightmark}
\fancyhead[L]{\warning{DRAFT}}
\fancyhead[R]{\warning{DEBUG ON}}

\fancyfoot[L]{\warning{DRAFT}}
\fancyfoot[R]{\warning{DEBUG ON}}

\fancyfoot[C]{\thepage}

\allowdisplaybreaks

\usepackage{chngcntr}
\counterwithin{figure}{section}


\newcommand{\figplaceholder}[1]{\ifdebug
	\begin{figure}[H]
		\begin{center}	
			\rule{8cm}{8cm}
			\caption{\color{red}placeholder}
			\label{fig:#1}
		\end{center}
	\end{figure}
\else
\PackageError{}{NO FIGURE}{}
\fi
}


\usepackage[acronym]{glossaries}\makeglossaries
\newcommand{\acr}[3]{\newacronym{#1}{#2}{#3}}
\newcommand{\symbl}[3]{\newglossaryentry{#1}{name = #2,	description = #3,}}



\usepackage[acronym]{glossaries}\makeglossaries
\newcommand{\acr}[3]{\newacronym{#1}{#2}{#3}}
\newcommand{\symbl}[3]{\newglossaryentry{#1}{name = #2,	description = #3,}}
\acr{AUT}{AUT}{AUTomatic control}
\acr{IETR}{IETR}{Institut d'Électronique et de Télécommunications de Rennes}
%
\acr{JPL}{JPL}{NASA Jet Propulsion Lab - Caltech - USA}
\acr{USP}{USP}{Universidade de São Paulo - Brasil}
\acr{Polimi}{POLIMI}{Politecnico Milano - Italia}

\acr{DPL}{DPL}{DIgSILENT Programming Language}
\acr{EDF}{EDF}{Électricité de France}
\acr{Enedis}{Enedis}{l'ancienne ERDF}


\acr{GUI}{GUI}{Interface Graphique d'utilisateur}

\acr{RMS}{RMS}{Root Mean Square - Moyenne quadratique}
\acr{EMT}{EMT}{Electro Magnetic Transient}

\debugtrue


\makeindex



\begin{document}
%\symbl{teste2}{$\beta$}{beta}
%\acr{Supelec}{Supélec}{École Supérieure d'Électricité}

\large
\renewcommand{\figurename}{Figure}


\begin{titlepage}
\begin{center}
% Upper part of the page. The '~' is needed because \\
% only works if a paragraph has started.
\includegraphics[width=60mm]{logos/supelec.jpeg}%~\\[0.5cm]
\vspace{50pt}
% Title
\rule{\linewidth}{0.5mm} \\[0.4cm]
{ \huge \bfseries \title \\[0.4cm] }
\rule{\linewidth}{0.5mm} \\[0.5cm]
\textsc{\Large \subtitle}\\[1.5cm]
\vspace{60pt}
% Author and supervisor
\begin{minipage}{0.4\textwidth}
\center
\normalsize
{Rafael Accácio NOGUEIRA\vspace{50pt} }

\large
\textbf{Orienté par\\M. Hervé GUÉGUEN}
 \\
\end{minipage}
\vfill
% Bottom of the page
{\large \today}
\end{center}

\teacher{M. Hervé GUÉGUEN}
\end{titlepage}
\conteudo
\newpage
\printglossary[type=\acronymtype,title=Liste des Acronymes]
\phantomsection\addcontentsline{toc}{section}{Liste des Acronymes}
\newpage 
\printglossary[title=Glossaire]
\phantomsection\addcontentsline{toc}{section}{Glossaire}
\newpage
%\section*{Remerciements}         % ne pas numéroter
\phantomsection\addcontentsline{toc}{section}{Remerciements} % inclure dans TdM

À Dieu, mes parents, ma petite amie 

\mysection{Objectif}
L'objectif de ce document est faire un rapport du stage, qui détaille le lieu de travail, toutes les choses produits pendant le stage en expliquant les méthodes utilisés et qui montre les résultats obtenus, les commente et dit les implications des conclusions.
\mysection{Introduction Générale}
Afin de compléter la formation du 2A CentraleSupélec, un stage a été réalisé entre lesm ois de juillet et septembre de 2017, au sein du laboratoire de l'équipe \gls{AUT} du	
\gls{IETR} en travaillant au cadre du projet d'\title, avec la orientation de M. Hervé GUÉGUEN.


\mysubsection{Sur le lieu de travail} 
\mysubsubsection{L'IETR}
L'IETR est un institut de recherche français, spécialisé en électronique et télécommunication, localisé à Rennes, comptant avec plus que 300 enseignants-chercheurs, ingénieurs, doctorants et administratifs, il est formé par équipes de recherche des écoles et instituts de recherche de la région comme le CNRS, l'Université Rennes 1, INSA de Rennes, CentraleSupélec et Université de Nantes.

En relation aux partenariats avec autres instituts et entreprises, L'IETR a une liste considérable de partenaires, incluent des centres publiques comme CEA, CNES et Club Automatique et Automatisation industrielle de la SEE, petites et moyennes entreprises privés comme A\&P Lithos, Adlightec et Advansee, et grandes groupes comme Alstom, EDF et Mitsubishi. 

Son partenariat International compte sur plus de 70 Universités, Instituts et Agences de recherche parmi tout le monde, incluent \gls{JPL}, \gls{Polimi} et \gls{USP}.

\mysubsubsection{Division des Équipes de Recherche}
Afin de meilleur catégoriser les thématiques des projets de recherche L'IETR est divisé en 6 départements/équipes:

\begin{itemize}
	\item Antennes \& Dispositifs Hyperfréquences (ADH)
	\item Signal \& Communications (SC)
	\item Ondes \& Signaux (OS)
	\item Image
	\item Microélectronique \& Microcapteurs (MM)
	\item Automatique (AUT)
\end{itemize}

L'organigramme structurel avec tant les parties de recherche quant les parties administratifs du IETR peut être vu dans la figure \ref{fig:organigramme_IETR_160717_v28}.
\pagebreak

\mysubsubsection{L'équipe AUT}
L'équipe de Automatique est basé a CentraleSupélec et travaille dans diverses thématiques utilisant les connaissances des domaines de analyse et commande des systèmes hybrides.

Ses projets ont des applications que couvrent diverses métiers, on peut voir quelques exemples de projets: comme dans le métier des bâtiments intelligents ( régulation de chauffage pour augmenter l'efficience énergétique )  ,  de la santé ( models des pancréas artificiels pour contrôler le métabolisme de glucose ), du transport ( stratégies de régulation de vitesse des automobiles) , de la distribution d'énergie ( Études de stabilisation de tension de générateurs distribués dans un réseau de distribution ) entre autres.

\vspace{.5cm}
\begin{figure}[H]
	\begin{center}	
		\includegraphics[width=\textwidth]{./organigramme_IETR_160717_v28.pdf}
		\caption{Organigramme du IETR.}
		\label{fig:organigramme_IETR_160717_v28}
	\end{center}
\end{figure}













\mysection{Le Projet}
L'objectif du projet réalisé pendant le stage était faire des simulations d'un réseaux de distribution électrique et développer des régulateur des tension des bus du circuit tout ça utilisant les logiciels MATLAB et DIgSILENT PowerFactory.

Les résultats obtenues a partir de ces simulations ont été interprété et comparé avec les résultats provenant du travaille de Marjorie Cosson \cite{cosson:tel-01374469} a fin d'une revalidation de ses conclusions.   

\mysection{Division du travail}
Pour faire le travail un peu plus simple, il était divisé en plusieurs parties:

\mysubsection{Première Partie}
La première partie consistait en lire le article de WAN Yidong \cite{yidong}, qu'explique d'une façon un peu simplifié le problème et montre des façons de calculer les gains entre la tension des bus et la puissance reactive des charges, en utilisant scripts écrits en \gls{DPL} dans le logiciel PowerFactory et la création d'une matrix de gain.

Après la lecture du article, la lecture de quelques parties de la thèse de Marjorie Cosson \cite{cosson:tel-01374469} a fin de comprendre le problème un peu meilleur   proposé et les résultats trouvés. 

\mysubsection{Deuxième Partie}

\mysection{Méthodologie}
Cette section a comme but expliquer les méthodes utilisées pour faire chaque tâche des parties 2 a 4 du travail, décrits dans les sections \ref{Deuxième Partie - Mise en Main} a \ref{Quatrième Partie - Intégration}.

\newcommand{\powerfactory}{DIgSILENT PowerFactory}
\mysubsection{Deuxième Partie - Mise en Main}
Comme dit, le logiciel \powerfactory a été utilisé pour faire des simulations du réseau, donc une petit explication du réseau e du logiciel sera fait dans cette section.

\mysubsubsection{À propos du \powerfactory}
Le \powerfactory est un logiciel beaucoup utilisé dans le métier d'Énergie, par entreprises comme \gls{EDF}, \gls{Enedis} pour faire des simulations des réseau électriques, que les permettent de vérifier stabilité en cas de panne et surcharge ou sous-charge de parties du réseau, calculer coûts d'opération et même programmer autres changements futur du réseau. Quelques autres instituts comme L'\gls{IETR} et \gls{Polimi}, par exemple l'utilise pour ses thèmes des thèses et autres recherches.   

\begin{figure}[H]
	\begin{center}	
		\includegraphics[width=\textwidth]{Methodologie/partie_2/gui_powerfactory_num.JPG}
		\caption{\gls{GUI} du \powerfactory.}
		\label{fig:gui_powerfactory}
	\end{center}
\end{figure}

\pagebreak
En peut voir dans la figure \ref{fig:gui_powerfactory} quelques panneaux basiques importants du \gls{GUI} du \powerfactory

\begin{enumerate}[I]
	\item Panneau Graphique\\ Où les diagrammes sont affichés, tant réseaux quant graphiques.
	\item Panneau des Outil\\ Où sont concentrés les outils du programme, pour modifier l'affichage, donnant une couleur différente par chaque bus par exemple; faire des diverses types de simulation,  de Court-circuit, calculs de flux de charge simulation \gls{RMS} et \gls{EMT}, etc.
	\item Panneau de Dessin\\ Outils pour dessiner des éléments du réseau, comme bus, transformateur, charge etc
	\item Panneau de Sortie\\ Où sont montrés les résultats des calculs et simulations, les avertissement et les erreurs.
\end{enumerate}


\mysubsection{À propos du réseau}
\newcommand{\trafoi}{40 MVA132/20}
\newcommand{\trafoii}{0.25MVA 20kV/0.4}
\newcommand{\trafoiii}{0.4MVA 20kV/0.4}
\newcommand{\trafoiv}{0.63MVA 20kV/0.4}
\newcommand{\cablei}{ARG7H1RX 185mmq}
\newcommand{\cableii}{ARG7H1RX 70mmq}
\newcommand{\cableiii}{Aerea Cu 70mmq}
Comme était dit, la figure \ref{fig:Diagramme_du_reseaux} démontre le réseau utilisé. Si on compare avec le réseaux utilisé en \cite{cosson:tel-01374469} et \cite{mariani2013controllo}, il est possible de voir que le réseaux de la figure  \ref{fig:Diagramme_du_reseaux} représente juste la moitié du original. Cette choix de prendre une partie du réseaux a comme raison la diminution des éléments et conséquemment la complexité des résolutions des simulations.

Le réseau est formé pour 16 charges et 3 générateurs distribués, 12 transformateurs.
Dans le réseau original deux des trois générateurs étaient des machines synchrones mais elles ont été remplacé par des panneaux photovoltaïque, a fin de faire les réponses des tests plus vite, en vue de la dynamique des panneaux considérablement plus vite que des machines synchrones, qui dépendent des constants électromécanique. 

A fin de faire une meilleur description du réseaux les tables \ref{tab:generateurs_du_reseaux} à \ref{tab:Charges} ont les caractéristiques des éléments qui la composent.

\hspace{-1.7cm}
\begin{minipage}{.5\textwidth}
\begin{table}[H]
	\captionsetup{justification=centering,margin=1cm}
	\caption{Générateurs Distribués du Réseau.}
	\label{tab:generateurs_du_reseaux}
	\centering
	\begin{tabular}{m{1cm}m{1.5cm}m{1.5cm}}
		\hline
		GD&P[MW] nominal&P[MW] 1p.m.\\
		\hline\\
		GD4&3.2&2.056124\\
		GD5&5.5&4.94595\\
		GD6&5.5&3.245381\\
		\hline\\
	\end{tabular}
\end{table}	
\end{minipage}
\begin{minipage}{.5\textwidth}
\begin{table}[H]
	\captionsetup{justification=centering}
	\caption{Transformateurs HV/MV.}
	\label{tab:Transformateurs_HV/MV}
	\centering
	\begin{tabular}{lc}
		\hline
		Model&40 MVA132/20\\
		\hline\\
		Puissance&50MVA\\
		Pertes Cuivre&176kW\\
		Tension de court-circuit Relative&15.5\%\\
		Taps&12\\
		Tension per Tap&1.5\%\\
		\hline\\
	\end{tabular}
\end{table}	
\end{minipage}
\begin{table}[H]
	\captionsetup{justification=centering,margin=2cm}
	\caption{Transformateurs MV/LV.}
	\label{tab:Transformateurs_MV/LV}
	\centering
	\begin{tabular}{lm{2cm}m{2cm}m{2cm}}
		\hline
		Modèle&0.25MVA 20kV/0.4&0.4MVA 20kV/0.4&0.63MVA 20kV/0.4\\
		\hline\\
		Puissance&250kVA&400kVA&630kVA\\
		Pertes Cuivre&2.6kW&3.7 kW&5.6kW\\
		Tension de court-circuit Relative&4\%&4\%&4\%\\
		Nombres de Transformateurs&1&6&4\\
		\hline\\
	\end{tabular}
\end{table}	
\vspace{-.5cm}
\begin{table}[H]
	\captionsetup{justification=centering,margin=2cm}
	\caption{Transformateurs.}
	\label{tab:Transformateurs}
	\centering
	\begin{tabular}{cc}
		\hline
		Nom&Modèle\\
		\hline\\
		TR AT/MT&\trafoi\\
		TR 2.19&\trafoiv\\
		TR 2.20&\trafoiii\\
		TR 2.21&\trafoiii\\
		TR 2.24&\trafoiii\\
		TR 2.25&\trafoiii\\
		TR 2.27.1&\trafoiv\\
		TR 2.27.3&\trafoiv\\
		TR 2.28&\trafoii\\
		TR 2.30&\trafoiv\\
		TR 2.31&\trafoiii\\
		TR 2.32&\trafoiii\\
		\hline\\
	\end{tabular}
\end{table}	
\vspace{-.5cm}
\begin{table}[H]
	\captionsetup{justification=centering,margin=2cm}
	\caption{Caractéristiques des Lignes.}
	\label{tab:Caractéristiques_des_Lignes}
	\centering
	\begin{tabular}{ccc}
		\hline
		Nom&Genre&Longueur[$ km $]\\
		\hline\\
		D2-02\_19&\cablei&3.6\\
		D2-19\_20	&\cablei&3.304\\
		D2-20\_21	&\cableiii&2.4\\
		D2-21\_22	&\cableiii&3.6\\
		D2-22\_23	&\cableiii&3\\
		D2-22\_28	&\cableii&2.4\\
		D2-23\_24	&\cableiii&3.08\\
		D2-24\_25	&\cableiii&1.65\\
		D2-25\_26	&\cableiii&1.8\\
		D2-26\_27	&\cableiii&2.2\\
		D2-28\_29	&\cableii&2.2\\
		D2-29\_30	&\cableii&2.4\\
		D2-30\_31	&\cableii&2.6\\
		D2-31\_32	&\cableii&2.7\\
		\hline\\
	\end{tabular}
\end{table}
\vspace{-.5cm}
\begin{table}[H]
	\captionsetup{justification=centering,margin=2cm}
	\caption{Lignes.}
	\label{tab:Lignes}
	\centering
	\begin{tabular}{llccccc}
		\hline
		Nom&Genre&Section[$ mm^2 $]&R[$ \Omega/km $]&L[$ mH/km $]&C[$ \mu F/km $]\\
		\hline\\
		\cablei&Câble&185&0.2180&0.350&0.2900\\
		\cableii&Câble&70&0.5800&0.41&0.2100\\
		\cableiii&Aérien&70&0.2681&1.286&0.0090\\
		\hline\\
	\end{tabular}
\end{table}	
\pagebreak
\begin{table}[H]
	\captionsetup{justification=centering,margin=2cm}
	\caption{Charges.}
	\label{tab:Charges}
	\centering
	\begin{tabular}{cccc}
		\hline
		Nom&Genre&P[$ MW $]1p.m.&Q[$MVAR$]1p.m.\\
		\hline\\
		C 2-19 &LV&0.1894&0.1265088\\
		C 2-20 &LV&0.1147&0.0774413\\
		C 2-21 &LV&0.1155&0.0782289\\
		C 2-23&MV&0&0.1\\
		C 2-24 &LV&0.1094&0.0741473\\
		C 2-25 &LV&0.1450&0.0984401\\
		C 2-26&MV&0.3993&0.2049369\\
		C 2-27.1&LV&0.2471&0.1656134\\
		C 2-27.2& MV&.6083&0.2971269\\
		C 2-27.3& LV&0.2094&0.1407233\\
		C 2-28 &LV&0.1205&0.08741\\
		C 2-29 &MV&0.1561&0.0798601\\
		C 2-30 &LV&0.1934&0.1347733\\
		C 2-31 &LV&0.0934&0.0640347\\
		C 2-32.1 &LV&0.1333&0.0923274\\
		C 2-32.2 &MV&0.5634&0.2791258\\
		\hline\\
	\end{tabular}
\end{table}	

\mysection{Troisième Partie - Programmation}

Pendant ce partie diverses scripts ont été crées en utilisant les langages MATLAB et Python, en intégration des bibliothèques Python du \powerfactory, qui pourvoit quelques fonctions pour accéder aux éléments du réseau, changer ses caractéristiques et faire des mesures de courant, tension etc.

Dans cette section chaque script est décrit par son fonctionnement, pour une description minutieuse, les codes sont présents dans l'appendice \ref{Programmes}. En ayant en tête que toutes fichiers utilisés sont disponibles dans le site \url{https://github.com/Accacio/scripts_powerfactory}.



\begin{enumerate}[\bfseries 5.1]
	\item $\mathbf{gain\_calc-load2bus.py}$\\
	\\Ce script change les valeurs de puissance réactive demandé par les charges pendant un intervalle de temps et prend les valeurs de tension du bus auquel la charge est connecté pendant ce temps et calcule le gain entre chaque charge et bus, formant une matrice de dimension $ 16\times16 $, avec le format suivant:
	\\	\begin{table}[H]\tiny
		\captionsetup{justification=centering,margin=0cm}
		\caption{Matrice des Gain de sortie du script $gain\_calc-load2bus.py$.}
		\centering
		\resizebox{\textwidth}{!}{\begin{tabular}{cccccc}
				$\frac{V_{N01}}{Q_{C 2-19}}$& $\frac{V_{N02}}{Q_{C 2-19}}$& $\frac{V_{N19}}{Q_{C 2-19}}$& $\frac{V_{N20}}{Q_{C 2-19}}$&$ \cdots $&$\frac{V_{N32}}{Q_{C 2-19}}$\\
				&&&&&\\
				$ \vdots $&$ \vdots $&$ \vdots $&$ \vdots $&$ \ddots $&$ \vdots $\\
%				$\frac{V_{N01}}{Q_{C 2-20}}$& $\frac{V_{N02}}{Q_{C 2-20}}$& $\frac{V_{N19}}{Q_{C 2-20}}$& $\frac{V_{N20}}{Q_{C 2-20}}$&$\cdots$&$\frac{V_{N32}}{Q_{C 2-20}}$\\
%				&&&&&\\
%				$\frac{V_{N01}}{Q_{C 2-21}}$& $\frac{V_{N02}}{Q_{C 2-21}}$& $\frac{V_{N19}}{Q_{C 2-21}}$& $\frac{V_{N20}}{Q_{C 2-21}}$&$\cdots$&$\frac{V_{N32}}{Q_{C 2-21}}$\\
%				&&&&&\\
%				$\frac{V_{N01}}{Q_{C 2-22}}$& $\frac{V_{N02}}{Q_{C 2-22}}$& $\frac{V_{N19}}{Q_{C 2-22}}$& $\frac{V_{N20}}{Q_{C 2-22}}$&$\cdots$&$\frac{V_{N32}}{Q_{C 2-22}}$\\
%				&&&&&\\
%				$\frac{V_{N01}}{Q_{C 2-24}}$& $\frac{V_{N02}}{Q_{C 2-24}}$& $\frac{V_{N19}}{Q_{C 2-24}}$& $\frac{V_{N20}}{Q_{C 2-24}}$&$\cdots$&$\frac{V_{N32}}{Q_{C 2-24}}$\\
%				&&&&&\\
%				$\frac{V_{N01}}{Q_{C 2-25}}$& $\frac{V_{N02}}{Q_{C 2-25}}$& $\frac{V_{N19}}{Q_{C 2-25}}$& $\frac{V_{N20}}{Q_{C 2-25}}$&$\cdots$&$\frac{V_{N32}}{Q_{C 2-25}}$\\
%				&&&&&\\
%				$\frac{V_{N01}}{Q_{C 2-26}}$& $\frac{V_{N02}}{Q_{C 2-26}}$& $\frac{V_{N19}}{Q_{C 2-26}}$& $\frac{V_{N20}}{Q_{C 2-26}}$&$\cdots$&$\frac{V_{N32}}{Q_{C 2-26}}$\\
				&&&&&\\
				$\frac{V_{N01}}{Q_{C 2-27.1}}$& $\frac{V_{N02}}{Q_{C 2-27.1}}$& $\frac{V_{N19}}{Q_{C 2-27.1}}$& $\frac{V_{N20}}{Q_{C 2-27.1}}$&$ \cdots $&$\frac{V_{N32}}{Q_{C 2-27.1}}$\\
				&&&&&\\
				$\frac{V_{N01}}{Q_{C 2-27.2}}$& $\frac{V_{N02}}{Q_{C 2-27.2}}$& $\frac{V_{N19}}{Q_{C 2-27.2}}$& $\frac{V_{N20}}{Q_{C 2-27.2}}$&$ \cdots $&$\frac{V_{N32}}{Q_{C 2-27.2}}$\\
				&&&&&\\				
				$\frac{V_{N01}}{Q_{C 2-27.3}}$& $\frac{V_{N02}}{Q_{C 2-27.3}}$& $\frac{V_{N19}}{Q_{C 2-27.3}}$& $\frac{V_{N20}}{Q_{C 2-27.3}}$&$ \cdots$&$\frac{V_{N32}}{Q_{C 2-27.3}}$\\
				&&&&&\\
				$\frac{V_{N01}}{Q_{C 2-28}}$& $\frac{V_{N02}}{Q_{C 2-28}}$& $\frac{V_{N19}}{Q_{C 2-28}}$& $\frac{V_{N20}}{Q_{C 2-28}}$&$\cdots$&$\frac{V_{N32}}{Q_{C 2-28}}$\\
				&&&&&\\
%				$\frac{V_{N01}}{Q_{C 2-29}}$& $\frac{V_{N02}}{Q_{C 2-29}}$& $\frac{V_{N19}}{Q_{C 2-29}}$& $\frac{V_{N20}}{Q_{C 2-29}}$&$\cdots$&$\frac{V_{N32}}{Q_{C 2-29}}$\\
%				&&&&&\\
%				$\frac{V_{N01}}{Q_{C 2-30}}$& $\frac{V_{N02}}{Q_{C 2-30}}$& $\frac{V_{N19}}{Q_{C 2-30}}$& $\frac{V_{N20}}{Q_{C 2-30}}$&$\cdots$&$\frac{V_{N32}}{Q_{C 2-30}}$\\
%				&&&&&\\
%				$\frac{V_{N01}}{Q_{C 2-31}}$& $\frac{V_{N02}}{Q_{C 2-31}}$& $\frac{V_{N19}}{Q_{C 2-31}}$& $\frac{V_{N20}}{Q_{C 2-31}}$&$\cdots$&$\frac{V_{N32}}{Q_{C 2-31}}$\\
%				&&&&&\\
				$ \vdots $&$ \vdots $&$ \vdots $&$ \vdots $&$ \ddots $&$ \vdots $\\
				&&&&&\\
				$\frac{V_{N01}}{Q_{C 2-32.1}}$& $\frac{V_{N02}}{Q_{C 2-32.1}}$& $\frac{V_{N19}}{Q_{C 2-32.1}}$& $\frac{V_{N20}}{Q_{C 2-32.1}}$&$\cdots$&$\frac{V_{N32}}{Q_{C 2-32.1}}$\\
				&&&&&\\
				$\frac{V_{N01}}{Q_{C 2-32.2}}$& $\frac{V_{N02}}{Q_{C 2-32.2}}$& $\frac{V_{N19}}{Q_{C 2-32.2}}$& $\frac{V_{N20}}{Q_{C 2-32.2}}$&$\cdots$&$\frac{V_{N32}}{Q_{C 2-32.2}}$\\
		\end{tabular}}		
	\end{table}
	Où \gls{Vnxx}  et \gls{Qcxx} sont la tension du bus $ Nxx $ e la puissance réactive de la charge $ Cx-xx $.
	
	\item $\mathbf{gain\_calc-generator2bus-test\_1.py}$\\
	\\Ce Script change les valeurs de puissance réactive des générateurs pendant un intervalle de temps e prends les valeurs de tension du bus auquel le générateur est connecté pendant ce temps et calcule le gain entre chaque générateur et bus, formant une matrice de dimension $ 3\times3 $. L'allure de la courbe utilisé peut être vu dans la figure \ref{fig:gaincalcgenerator2bustest1}. Le format de la matric est montré dans la table \ref{tab:gaincalcgenerator2bustest1}.
	\begin{figure}[H]
		\begin{center}	
			\includegraphics[width=8cm]{Methodologie/partie_3/gain_calc-generator2bus-test_1.pdf}
			\caption{Allure de la courbe utilisé pendant le Test 1.}
			\label{fig:gaincalcgenerator2bustest1}
		\end{center}
	\end{figure}
\begin{table}[H]
	\captionsetup{justification=centering,margin=2cm}
	\caption{Matrice des Gain de sortie du script $gain\_calc-generator2bus-test\_1.py$.}
	\label{tab:gaincalcgenerator2bustest1}
	\centering
	\begin{tabular}{ccc}
		$ \frac{V_{N21}}{Q_{GD4}} $&$ \frac{V_{N29}}{Q_{GD4}} $&$ \frac{V_{N23}}{Q_{GD4}} $\\
		&&\\
		$ \frac{V_{N21}}{Q_{GD5}} $&$ \frac{V_{N29}}{Q_{GD5}} $&$ \frac{V_{N23}}{Q_{GD5}} $\\
		&&\\
		$ \frac{V_{N21}}{Q_{GD6}} $&$ \frac{V_{N29}}{Q_{GD6}} $&$ \frac{V_{N23}}{Q_{GD6}} $\\
	\end{tabular}
\end{table}
Où \gls{Qgdx} est la puissance réactive du générateur $ GDx $.

	\item $\mathbf{gain\_calc-generator2bus-test\_2.py}$\\
	\\Fondamentalement le même que $gain\_calc-generator2bus-test\_1.py$, mais la courbe de changement de puissance suivre l'allure de la figure \ref{fig:gaincalcgenerator2bustest2}. Le format de la matrice est le même montré dans la table \ref{tab:gaincalcgenerator2bustest1}.
	\begin{figure}[H]
		\begin{center}	
			\includegraphics[width=8cm]{Methodologie/partie_3/gain_calc-generator2bus-test_2.pdf}
			\caption{Allure de la courbe utilisé pendant le Test 2.}
			\label{fig:gaincalcgenerator2bustest2}
		\end{center}
	\end{figure}
	\item $\mathbf{import\_generator\_csv.py}$\\
	\\Ce script prend des fichiers \gls{CSV} qui a données de puissance nominal, active et réactive des générateurs, et les charge en chaque générateur.
	\begin{figure}[H]
		\begin{center}	
			\includegraphics[width=8cm]{Methodologie/partie_3/csv_sample.JPG}
			\caption{Exemple d'un fichier CSV, séparé par point-virgule.}
			\label{fig:csv_sample}
		\end{center}
	\end{figure}
	La choix d'utiliser fichiers \gls{CSV} a été fait en raison de ce type de fichier est texte sans brut sans formatage, sa taille est très réduite. Et ce type de fichier est très simple pour faire des modifications et création a partir d'une table excel, ainsi que la conversion entre ces deux extensions.  
	\\
	\item $\mathbf{import\_load\_csv.py}$\\
	\\Ce script prend des fichiers \gls{CSV} qui a données de puissance active et réactive des charges, et les charge en chaque respective charge.
	\\
	\item $\mathbf{csv2mat.m}$\\
	\\Ce script fait en MATLAB prend un fichier \gls{CSV} sortie de \todo{simulation} et le transforme en un fichier .mat. Il crée une structure composés par autres structures qui sont les éléments du circuit et composées par ses données  pendant le temps, a fin de les utiliser pour faire des graphiques.
	\\
	\item $\mathbf{printalldata.m et printpdf.m}$\\
	\\Ces scripts ont été créés pour faire les plots des graphiques et les sauvegarder pour un futur use dans les rapports.  
\end{enumerate}




































\section{Résultats}
\begin{itemize}
	\item 
\end{itemize}

\newpage
\mysection{Difficultés}
Pendant le projet quelques difficultés ont été trouvées et les principalles sont les suivantes:

\begin{itemize}
	\item Long temps de calcul\\
	Le modèle avec le régulateur implémenté dans simulink prend 5 min a peu près, ce qui conduit à des difficultés de \textit{debuging} du système et de génération des résultats.\\ 
	\item Touts blocs en PowerFactory sont synchrones\\
	Dû a ça, le filtre a été implémenté dehors simulink, augmentant la complexité du modèle, 4 blocs plus ont été crées.\\
	\item Conditions Initiales\\ 
	A cause des différents blocs utilisés chaque bloc avait besoin d'avoir ses conditions initiales cohérentes entre elles,ce qui causait de problèmes car elles ne fussent pas cohérentes.\\
	\item Documentation du \powerfactory\\
	La documentation du logiciel n'est pas bien détaillée, causant quelquefois ambiguïté, créant le besoin de chercher l'information en autres lieux.\\
	\item Communauté PowerFactory presque inexistant\\
	Pour trouver des informations il fallait chercher a l'internet, mais la principale source était le faq du PowerFactory, qu'est aussi pauvre que la documentation. 
\end{itemize}

\mysection{Prochains Travails}
Pour prochains travails quelques choses peuvent être suggérées:
\begin{itemize}
	\item Implémenter un bloc dans PowerFactory qui soit appelé en temps différent du pas de la simulation.
	\item Réduire temps de communication PowerFactory$ \leftrightarrows $Matlab
	\item Automatiser les diverses simulations et la génération des figures.
\end{itemize}
\newpage
\mysection{Conclusions}
On peut arriver a quelques conclusions après avoir vu toute le méthodologie utilisée et les résultats en graphiques et tableaux, et elles sont:

\begin{itemize}
	\item \textbf{Scripting en PowerFactory}\\
	Il est possible d'utiliser Python et la combinaison entre l'API PowerFactory avec les bibliothèques Python déjà crées permet une gamme de possibilités.\\
	\item \textbf{Intégration Matlab $\mathbf{\leftrightarrows }$ PowerFactory}\\
	Assez facile si toute les variables du système sont préalablement connues.\\
	\item \textbf{Stabilité du système dépendent de la valeur de $ a $}\\
	Régulateur fonctionne mais dépend de $ a $ pour rester stable, valeurs plus proches de 1 font le système stabiliser plus vite.
	
\end{itemize}


\newpage
\begin{appendices}
%	\input{Appendices/}
\end{appendices}
\newpage
\listoftables
\phantomsection\addcontentsline{toc}{section}{Liste des tableaux} %
\newpage
\listoffigures
\phantomsection\addcontentsline{toc}{section}{Table des figures}
\newpage
\bibliographystyle{plain}
\bibliography{bibliografia}

\end{document}