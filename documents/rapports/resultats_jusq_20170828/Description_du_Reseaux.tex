\mysection{Description du Réseaux}
Comme était dit, la figure \ref{fig:Diagramme_du_reseaux} démontre le réseaux utilisé. On peut voir que le réseaux est formé pour 16 charges et 3 générateurs distribués, 12 transformateurs.
Dans le réseaux original les générateurs étaient des machines synchrones mais elles ont été remplacé par des panneaux photovoltaïque, a fin de faire les réponses des tests plus vite, en vue de la dynamique des panneaux considérablement plus vite que des machines synchrones. 

\begin{table}[H]
	\captionsetup{justification=centering,margin=2cm}
	\caption{Générateurs Distribués du Réseaux }
	\centering
	\begin{tabular}{m{1cm}m{1.5cm}m{1.5cm}}
	\hline
	GD&P[MW] nominal&P[MW] 1p.m.\\
	\hline\\
	GD4&3.2&2.056124\\
	GD5&5.5&4.94595\\
	GD6&5.5&3.245381\\
	\hline\\
	\end{tabular}
\end{table}	

\begin{table}[H]
	\captionsetup{justification=centering,margin=2cm}
	\caption{Transformateurs HV/MV }
	\centering
	\begin{tabular}{lc}
		\hline
		Model&40 MVA132/20\\
		\hline\\
		Puissance&50MVA\\
		Pertes Cuivre&176kW\\
		Tension de court-circuit Relative&15.5\%\\
		Taps&12\\
		Tension per Tap&1.5\%\\
		\hline\\
	\end{tabular}
\end{table}	

\begin{table}[H]
	\captionsetup{justification=centering,margin=2cm}
	\caption{Transformateurs MV/LV }
	\centering
	\begin{tabular}{lm{2cm}m{2cm}m{2cm}}
		\hline
		Modèle&0.25MVA 20kV/0.4&0.4MVA 20kV/0.4&0.63MVA 20kV/0.4\\
		\hline\\
		Puissance&250kVA&400kVA&630kVA\\
		Pertes Cuivre&2.6kW&3.7 kW&5.6kW\\
		Tension de court-circuit Relative&4\%&4\%&4\%\\
		Nombres de Transformateurs&1&6&4\\
		\hline\\
	\end{tabular}
\end{table}	

\newcommand{\trafoi}{40 MVA132/20}
\newcommand{\trafoii}{0.25MVA 20kV/0.4}
\newcommand{\trafoiii}{0.4MVA 20kV/0.4}
\newcommand{\trafoiv}{0.63MVA 20kV/0.4}
\begin{table}[H]
	\captionsetup{justification=centering,margin=2cm}
	\caption{Transformateurs}
	\centering
	\begin{tabular}{cc}
		\hline
		Nom&Modèle\\
		\hline\\
		TR AT/MT&\trafoi\\
		TR 2.19&\trafoiv\\
		TR 2.20&\trafoiii\\
		TR 2.21&\trafoiii\\
		TR 2.24&\trafoiii\\
		TR 2.25&\trafoiii\\
		TR 2.27.1&\trafoiv\\
		TR 2.27.3&\trafoiv\\
		TR 2.28&\trafoii\\
		TR 2.30&\trafoiv\\
		TR 2.31&\trafoiii\\
		TR 2.32&\trafoiii\\
		\hline\\
	\end{tabular}
\end{table}	

\newcommand{\cablei}{ARG7H1RX 185mmq}
\newcommand{\cableii}{ARG7H1RX 70mmq}
\newcommand{\cableiii}{Aerea Cu 70mmq}

\begin{table}[H]
	\captionsetup{justification=centering,margin=2cm}
	\caption{Lignes}
	\centering
	\begin{tabular}{llccccc}
		\hline
		Nom&Genre&Section[$ mm^2 $]&R[$ \Omega/km $]&L[$ mH/km $]&C[$ \mu F/km $]\\
		\hline\\
		\cablei&Câble&185&0.2180&.0350&0.2900\\
		\cableii&Câble&70&0.5800&0.41&0.2100\\
		\cableiii&Aérien&70&0.2681&1.286&0.0090\\
		\hline\\
	\end{tabular}
\end{table}	


\begin{table}[H]
	\captionsetup{justification=centering,margin=2cm}
	\caption{Caractéristiques des Lignes}
	\centering
	\begin{tabular}{ccc}
		\hline
		Nom&Genre&Longueur[$ km $]\\
		\hline\\
		D2-02\_19&\cablei&3.6\\
		D2-19\_20	&\cablei&3.304\\
		D2-20\_21	&\cableiii&2.4\\
		D2-21\_22	&\cableiii&3.6\\
		D2-22\_23	&\cableiii&3\\
		D2-22\_28	&\cableii&2.4\\
		D2-23\_24	&\cableiii&3.08\\
		D2-24\_25	&\cableiii&1.65\\
		D2-25\_26	&\cableiii&1.8\\
		D2-26\_27	&\cableiii&2.2\\
		D2-28\_29	&\cableii&2.2\\
		D2-29\_30	&\cableii&2.4\\
		D2-30\_31	&\cableii&2.6\\
		D2-31\_32	&\cableii&2.7\\
		\hline\\
	\end{tabular}
\end{table}	

\begin{table}[H]
	\captionsetup{justification=centering,margin=2cm}
	\caption{Charges}
	\centering
	\begin{tabular}{cccc}
		\hline
		Nom&Genre&P[$ MW $]1p.m.&Q[$MVAR$]1p.m.\\
		\hline\\
		C 2-19 &LV&0.1894&0.1265088\\
		C 2-20 &LV&0.1147&0.0774413\\
		C 2-21 &LV&0.1155&0.0782289\\
		C 2-23&MV&0&0.1\\
		C 2-24 &LV&0.1094&0.0741473\\
		C 2-25 &LV&0.1450&0.0984401\\
		C 2-26&MV&0.3993&0.2049369\\
		C 2-27.1&LV&0.2471&0.1656134\\
		C 2-27.2& MV&.6083&0.2971269\\
		C 2-27.3& LV&0.2094&0.1407233\\
		C 2-28 &LV&0.1205&0.08741\\
		C 2-29 &MV&0.1561&0.0798601\\
		C 2-30 &LV&0.1934&0.1347733\\
		C 2-31 &LV&0.0934&0.0640347\\
		C 2-32.1 &LV&0.1333&0.0923274\\
		C 2-32.2 &MV&0.5634&0.2791258\\
		\hline\\
	\end{tabular}
\end{table}	
