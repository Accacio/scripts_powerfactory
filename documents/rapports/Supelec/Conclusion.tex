
\newpage
\mysection{Difficultés}
Pendant le projet quelques difficultés ont été trouvé et les principaux sont les suivantes:

\begin{itemize}
	\item Long temps de calcule\\
	Le modèle avec le régulateur implémenté dans simulink prend 5 min a peu près, quoi a difficulté le \textit{debuging} du système comme la génération des résultats.\\ 
	\item Touts blocs en PowerFactory sont synchrones\\
	Dû a ça, le filtres a été implémenté dehors simulink, augmentant la complexité du modèle, 4 blocs plus ont été crée.\\
	\item Conditions Initiales\\ 
	A cause des différents blocs utilisés chaque bloc avait besoin d'avoir ses conditions initiales cohérents entre elles, qui causait de problèmes cas elles ne fussent pas cohérents.\\
	\item Documentation du \powerfactory\\
	Documentation du logiciel n'est pas bien détaillé, causant quelquefois ambiguïté, créant le besoin de chercher l'information en autres lieux.\\
	\item Communauté PowerFactory presque inexistant\\
	Pour trouver des informations il fallait chercher a l'internet, mais la principale source était le faq du PowerFactory, qu'est si réticent quant la documentation. 
\end{itemize}

\mysection{Prochains Travails}
Pour prochains travails quelques choses peuvent être suggérées:
\begin{itemize}
	\item Implémenter un bloc dans PowerFactory que soit appelé en temps différent du pas de la simulation.
	\item Réduire temps de communication PowerFactory$ \leftrightarrows $Matlab
	\item Automatiser les diverses simulations et la génération des figures.
\end{itemize}
\newpage
\mysection{Conclusions}
On peut arriver a quelques conclusions après avoir vu toute le méthodologie utilisé et les résultats en graphiques et tableaux, et elles sont:

\begin{itemize}
	\item \textbf{Scripting en PowerFactory}\\
	Il est possible d'utiliser Python et la combinaison entre l'API PowerFactory avec les bibliothèques Python déjà crées permet une gamme de possibilités.\\
	\item \textbf{Intégration Matlab $\mathbf{\leftrightarrows }$ PowerFactory}\\
	Assez facile si toute les variables du système sont préalablement connues.\\
	\item \textbf{Stabilité dus système dépendent de la valeur de $ a $}\\
	Régulateur fonctionne mais dépend de $ a $ pour rester stable, valeurs plus proches de 1 font le système stabiliser plus vite.
	
\end{itemize}